% !TEX root = ../proposal.tex
%
\pdfbookmark[0]{Abstract}{Abstract}
\chapter*{Abstract}
\label{sec:abstract}
\vspace*{-10mm}

Because an increasing amount of computer engineers have been frustrated with failing software products due to misinterpretation of their customers needs, \ac{EDSD} has gained a lot of traction during the past decade.
These experiments help the creators of a piece of software to test and validate their assumptions about a product and its features early in the development process.
User feedback is the data upon which these experiments rely.
While active user feedback, e.g. in form of surveys, can yield useful qualitative observations about the product, passive user feedback generally yields more quantitative and reliable data.
Passive user feedback can for example be generated by measuring the time a user needs to perform certain actions in the software.
The area of \acl{EDSD} in general is already well researched, but no one has as of yet researched how to design and implement a system for retrieving and analysing passive user feedback using event sourcing.
Event sourcing is a storage solution in which data is stored in the form of immutable event objects representing actions on business objects; in combination, all these events represent the application state.
Designing and implementing such a system includes researching solutions for storing, aggregating and analysing user feedback data.
A prototype using the chosen technologies is then implemented, which involves setting up and configuring various services as well as writing software that connects these services.
The system is evaluated via a controlled experiment in order to validate the assumptions in this thesis.
This will presumably involve either a user test for gathering passive user feedback, or analysing the user actions in an existing data set~\cite{Deka:2017:Rico}.
The gathered data is then aggregated and analysed using the implemented system, and then compared with the results of a questionnaire with usability experts.
