% !TEX root = ../proposal.tex
%
\pdfbookmark[0]{Abstract}{Abstract}
\chapter*{Abstract}
\label{sec:abstract}
\vspace*{-10mm}

Because an increasing amount of computer engineers have been frustrated with failing software products due to misinterpretation of their customers needs, the role of experimentation in software engineering has gained a lot of traction during the past decade.
%These experiments help the creators of a piece of software to test and validate their assumptions about a product and its features early in the development process.
%User feedback is the data upon which these experiments rely.
%While active user feedback, e.g. in form of surveys, can yield useful qualitative observations about the product, the fact that the user has to actively take time for giving their feedback is often problematic.
%Passive user feedback avoids this problem by automatically collecting feedback while the user works with the system; an example for this is measuring the time a user needs to perform certain actions in the software.
% What is the problem?
The area of innovation via experimentation in the software industry is already well researched, but lacks a more holistic view on collecting, aggregating and analysing the passive user feedback needed for evaluating an experiment.
%Why is this a problem?
This is problematic because, although many companies realise the need for experimentation in their development process, they struggle with the systematic adoption of these practices.
%What is the / a solution?
By researching how to design and implement a system for establishing passive user feedback with considerably low effort, this thesis aims to fill this gap.
It is expected that event sourcing is a good fit for realising such a system, which will be further investigated in the thesis.
%Event sourcing is a storage solution in which data is stored in the form of immutable event objects representing actions on business objects; in combination, all these events represent the application state.
Designing and implementing said system includes researching solutions for storing, aggregating and analysing user feedback data.
A prototype using the chosen technologies is then implemented, which involves setting up and configuring various services as well as writing software that connects thesm.
The system is evaluated via a controlled experiment in order to validate the assumptions in this thesis.
This will presumably involve either a user test for gathering passive user feedback, or analysing the user actions in an existing data set.
The gathered data is then aggregated and analysed using the implemented system, and then compared with the results of a questionnaire with usability experts.
%Why is this a solution / what is ?
Practitioners in the software industry could use the system presented in the thesis as a reference for introducing collection and analysis of passive user feedback into their software.
