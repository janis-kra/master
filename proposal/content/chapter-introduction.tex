% !TEX root = ../proposal.tex
%
\chapter{Introduction}
\label{sec:intro}

Best practices in software development changed dramatically since its origins in the 1950's~\cite{boehm2006view}.
Software engineering in most of the 20th century was largely based on careful planning and specification and therefore rather slow-paced and unflexible.
Since twenty years however, these plan-driven development methods have been frustrating many people because the circumstances change more and more rapidly -- not least due to the growing importance of the internet~\cite{Williams2003}.
A result of this rapid change of the environment was that product features, which were perfectly valid at the time of their envisioning, became outdated, irrelevant or otherwise undesirable during the planning or implementation phase.
This insight often only was gained when the feature was shipped to the customer.
As a consequence, short release cycles and fast customer feedback have steadily grown in importance.
These values are emphasised in the agile software development method~\cite{fowler2001agile} in general and in experimental software engineering in particular~\cite{???}.

The core premise of experimental software engineering is developing and validating MVPs of envisioned features in short sprints.
The results of such an experiment are then used for deciding wether the feature should become a part of the final product.
This approach is contrary to classical software development methods\todo{really 'methods' or sth. else?}, in which the stakeholders make certain assumptions which result in the envisioning of a feature, without consulting a customer first.
