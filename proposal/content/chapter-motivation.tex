% !TEX root = ../proposal.tex
%
\chapter{Motivation}
\label{sec:motivation}

Several companies have published their findings about \ac{EDSD} in the form of scientific papers or blog posts.
Amongst others, Microsoft~\cite{Kohavi2013}, LinkedIn~\cite{Xu2015}, Google~\cite{Tang2010}, Facebook~\cite{Bakshy2014}, and Netflix~\cite{WEB:Netflix:2016} issued reports about best practices and pitfalls that they experienced, all concluding that \ac{EDSD} benefits both the choice about which features to implement as well as their quality.
In addition, several authors have published scientific papers in which they propose models and architectures for experimental software development~\cite{Fagerholm2014,Fagerholm2017,Johanssen2017,Lindgren2015}.
Although practicioners in the software industry often claim that they embrace experimentation in general and continuous experimentation in particular, these practices are still not fully adopted according to a study by \citet{lindgren2015software}; especially the systematic and continuous aspects of experimentation lack adoption.

% Problem f�r Motivation ist hier: Wenn ich nicht direkt auf Event Sourcing Bezug nehme (d.h. es offen lasse ob Event Sourcing verwendet wird oder nicht) wird der Scope zu gro�: "Wie kann an eine Experimentier-Plattform implementieren" haben andere Leute schon beantwortet.
While the overall area of experimentation-driven software development is thus well researched, no one has as of today researched how to combine this with event sourcing.
This combination is believed to be a good fit for the following reasons:

\begin{enumerate}
\item Event replayability could be used to further mitigate the risk of experimentation: If an experimental change in the software causes some faulty event to be created, it can later be modified or deleted.
Contrary to classical relational databases, the overall application state in an event store would not be corrupted by this.
It would also be possible to create separate event stores just for the experiment, which could later be applied to the main event store by replaying the events from the experimental store.
\item When an experiment has to be analysed, temporal queries could be used for retrieving all events in the experiments time frame.
\item Event sourcing and \ac{CQRS} advocate a distributed system architecture -- which is also true for \ac{EDSD}~\cite{source?}.
\end{enumerate}

For these reasons, it is to be expected that an experimentation system using an event store for storage is a promising alternative to existing systems.
In particular, the temporal query feature could render additional aggregation services such as ElasticSearch obsolete.
Additional backup systems that roll back changes issued by an experiment could also possibly be removed because of an event stores inherent ability to roll back and redo changes.
These assumptions would have to be validated over the course of the actual thesis.


%(1) In particular, no one has as of today researched how to combine experimentation with event sourcing.
%(2) In particular, no one has as of today implemented a more distributed experimentation system.
%(3) In particular, research about concrete architectures, which are in addition validated by a user study, is slim.
%
% TODO: What exactly is the reason for starting this thesis at this point? (1) already implicates more than I thought I would assume at this point, but seems to be the best choice for now...
