% !TEX root = ../proposal.tex
%
\chapter{Motivation}
\label{sec:motivation}

Several companies have published their findings about experimentation-driven software development in the form of scientific papers or blog posts.
Amongst others, Microsoft~\cite{Kohavi2013}, LinkedIn~\cite{Xu2015}, Google~\cite{Tang2010}, Facebook~\cite{Bakshy2014}, and Netflix~\cite{WEB:Netflix:2016} issued reports about best practices and pitfalls that they experienced, all concluding that experimentation-driven software development benefits both the choice about which features to implement as well as their quality.
Also, several authors have published scientific papers in which they propose models and architectures for experimental software development~\cite{Fagerholm2014,Fagerholm2017,Johanssen2017,Lindgren2015}.
Although practicioners in the software industry often claim that they embrace experimentation in general and continuous experimentation in particular, these practices are still not fully adopted according to a study by \citet{lindgren2015software}; especially the systematic and continuous aspects of experimentation lack adoption.
















(1) In particular, no one has as of today researched how to combine experimentation with event sourcing.
(2) In particular, no one has as of today implemented a more distributed experimentation system.
(3) In particular, research about concrete architectures, which are in addition validated by a user study, is slim.

 TODO: What exactly is the reason for starting this thesis at this point? (1) already implicates more than I thought I would assume at this point, but seems to be the best choice for now...


 

