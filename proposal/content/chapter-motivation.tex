% !TEX root = ../proposal.tex
%
\chapter{Motivation}
\label{sec:motivation}

Several companies have published their findings about experimentation in the software development process, in the form of scientific papers or blog posts.
Amongst others, Microsoft~\cite{Kohavi2013}, LinkedIn~\cite{Xu2015}, Google~\cite{Tang2010}, Facebook~\cite{Bakshy2014}, and Netflix~\cite{WEB:Netflix:2016} issued reports about best practices and pitfalls that they experienced, all concluding that introducing experimentation into the software development process benefits both the choice about which features to implement as well as their quality.
In addition, several authors have published scientific papers in which they propose models and architectures for experimental software development~\cite{Fagerholm2014,Fagerholm2017,Johanssen2017,Lindgren2015}.
Although practicioners in the software industry often claim that they embrace experimentation in their research and development process, these practices are still not fully adopted according to a study by \citet{lindgren2015software}; especially the systematic and continuous aspects of experimentation lack adoption.

% Problem f�r Motivation ist hier: Wenn ich nicht direkt auf Event Sourcing Bezug nehme (d.h. es offen lasse ob Event Sourcing verwendet wird oder nicht) wird der Scope zu gro�: "Wie kann an eine Experimentier-Plattform implementieren" haben andere Leute schon beantwortet.
While the overall area of experimentation in software development is thus well researched, most papers are either more of a report about specific best practices or a rather abstract description of an architecture.
This is problematic for potential applicants of experimentation in software development, because while these reports are very practice-oriented, they lack a more general view on the problem at hand, and the other way around for the architecture-focused papers.
Therefore, the aim of this thesis is to bridge this gap and provide a more holistic view on collecting, aggregating and analysing passive user feedback.
This will enable applicants to implement their own innovation experiment systems based on a scientifically tested architecture.


%(1) In particular, no one has as of today researched how to combine experimentation with event sourcing.
%(2) In particular, no one has as of today implemented a more distributed experimentation system.
%(3) In particular, research about concrete architectures, which are in addition validated by a user study, is slim.
%
% TODO: What exactly is the reason for starting this thesis at this point? (1) already implicates more than I thought I would assume at this point, but seems to be the best choice for now...
