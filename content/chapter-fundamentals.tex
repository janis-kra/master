% !TEX root = ../thesis.tex
%
\chapter{Fundamentals}
\label{sec:fundamentals}

[insert fluff piece about fundamentals here]

\section{Experiment-driven Software Development}
\label{sec:fundamentals:edsd}

\cite{Olsson2012}

\section{Continuous experimentation surveys}
\label{sec:related:surveys}

\cite{lindgren2015software}

\cite{Bosch2012}

\cite{Gutbrod2017}

\section{Controlled Experiments}
\label{sec:fundamentals:experiments}

The earliest controlled experiment dates back to the 1700s, where the crew of a British ship suffered from scurvy, a common suffering that sailors tend to experience when exposed to a limited diet like on sea ships during that time.
As the ship was charging citrus fruits, the captain of the ship decided to do an experiment: 50\% of his sailors, the \emph{treatment} group, had limes added to their diet.
The other half of the crew did not eat any citrus fruits; these were the \emph{control} group.
Although the reasons were not known during that moment, the experiment was a success: The treatment group got better, and eventually all sailors had citrus fruits added to their diet~\cite{rossi2003evaluation,marks2000progress}.

This anecdote is an example for the simplest type of controlled experiment, in which users are randomly assigned to either the control or the treatment group.
While the treatment group is treated with the new version of the system that is being tested, the control group experiences the same system as before, without the modification.
Thus, each group contains 50\% of the user base, and the results of the control group can be used in order to control wether the treatment had any statistically significant effect~\cite{Kohavi2009}.
\todo{enough about controlled experiments?}
%The test that detects wether two variants are statistically different is called the \emph{null hypothesis}.

\section{Event sourcing \& CQRS}
\label{sec:fundamentals:event}

Event sourcing was first introduced by Greg Young in 200x~\cite{source/link}.
The core premise is that, contrary to classical relational databases, all data is stored in the form of immutable events, which represent the whole application state when aggregated.


\cite{WEB:Fowler:2005}

\cite{WEB:Fowler:2011}

\section{CQRS}
\label{sec:fundamentals:CQRS}
