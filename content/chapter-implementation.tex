% !TEX root = ../thesis.tex
%
\chapter{Implementation}
\label{ch:implementation}

This chapter describes details about the implementation and configuration of the passive user feedback system.
Appendix \cref{ch:appendix:source} contains the source code of the services and applications that were created over the course of this thesis; these are also available via GitHub\footnote{\url{https://github.com/janis-kra/master-impl}}\todo{link to "final" tag when done}.

\section{Service Orchestration}
\label{sec:implementation:orchestration}

All services run in their own Docker containers in order to facilitate distribution of the system.
Orchestration of the containers is done via \emph{Docker Compose} (short \emph{Compose}).
Compose is nice because reasons\todo{insert blurb about Compose}.

Docker allows to assign each container a custom network which allows for virtualization of distinct and separated networks even though the containers all on the same server.
This is done via the \texttt{networks} setting for all container definitions, in this case setting the network to \texttt{esnet}.
The \texttt{esnet} network is a default bridge network, thus ...

\section{Client Application}
\label{sec:implementation:client}

\section{Storage Configuration}
\label{sec:implementation:storage}

The storage layer of the passive user feedback system is realized via the event store reference implementation (\cref{sec:classifications:storage}).
An officially maintained Docker image\footnote{\url{https://hub.docker.com/r/eventstore/eventstore/}} is used as the starting point and requires little additional configuration.
The relevant lines from the \texttt{docker-compose.yml} file are listed in \todo{cref to code listing here}.

% code listing: docker-compose.yml#eventstore

This spins up a EventStore container with mostly default settings, which could be further configured via ?.
\todo{explain what could be configured+what the default settings are}

Aside from specifying the image base and its name, these settings make the EventStore's TCP and HTTP interfaces available via ports 1113 and 2113.
The container can be accessed by other containers via its name, \texttt{eventstore}.
Thus, the client application can post events via HTTP to \url{http://evenstore:2113}, while the bridge to the aggregation service (cf. \cref{sec:implementation:bridge}) listens to new events via the TCP address \url{tcp://eventstore:1113}.

\section{Storage-Aggregation Bridge}
\label{sec:implementation:bridge}

Because Elasticsearch does not have dedicated support for importing data from an event store, an additional service had to be implemented which serves as a bridge between the two services.

\section{Aggregation Service}
\label{sec:implementation:aggregation}

\section{Analysis Application}
\label{sec:implementation:analysis}

