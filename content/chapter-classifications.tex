% !TEX root = ../thesis.tex
%
\chapter{Classifications}
\label{sec:classifications}

\section{Storage Solutions}
\label{sec:system:storage}

The first technical decision to make is what to use as a storage solution.
I first explain requirements which the solution has to fulfill, then present the candidates, and ultimately evaluate according to these requirements which solution is most appropriate.

\subsection{Requirements}

\ac{CSE} requires the storage layer of such a system to have certain capabilities, namely flexibility and scalability.
In addition to basic storage capabilities, the storage solution has to provide some functionalities related to the use case (temporal queries).
I have identified the following requirements for the storage solution, based on which the candidates will be evaluated:

\begin{description}
\item [Basic storage capabilities]
The storage solution must at least provide means of creating and reading persisted data records.
\item [Modify]
In order to facilitate experimentation, the storage solution shall have some means of correcting faulty data records, i.e. an update and delete mechanic.
This is desirable in cases where a change in the software introduced erroneous behaviour which has to be undone.
%It would not be sufficient to just spin up a separate instance of the storage solution for the experiment because users should not be aware of whether they are part of the treatment group.
\item [Cascading update]
In addition to a regular modify feature, the storage solution ideally also has some means of correcting all data that is invalid due to being dependent on one invalid data record.
\item [Temporal queries]
It would presumably be useful to be able to retrieve all data records that were created within a specific time frame via some query.
These queries have to be efficient because running an analysis must not affect the storage solution's normal operation performance.
\todo{would this even be used with an aggregation service???}
\item [Vertical scalability]
For performance reasons, the storage solution shall be vertically scalable, i.e. scale \emph{up} when more computing power is added.
%This allows have some mechanism for handling large a large amount of requests on the same data.
\item [Horizontal scalability]
The storage solution shall also be able to scale horizontally, i.e. scale \emph{out}, which means that it is able to increase its performance linearly by adding more computing nodes to the network.
This is often solved via \emph{sharding}, where the data is split over multiple logical or physical storage instances, each of them shouldering a portion of the cumulative system load.
\todo{find a more precise definition for scalability, redo these requirements}
%Database clustering is done for various reasons. Clusters can improve availability, fault tolerance, and increase performance by applying a divide and conquer approach as work is distributed over many machines. Clustering is sometimes combined with partitioning and sharding to further break up a large complex task into smaller, more manageable units. 
\item [HTTP interface]
The storage solution shall have a HTTP interface, therefore allowing other services and applications to store and retrieve data via a universally supported medium.
Using HTTP comes with a performance overhead compared to other protocols such as TCP, but as  the advantages in terms of flexibility outweigh this drawback\todo{says who??}; in cases where performance is critical, TCP could still be used if supported.
It should be noted that almost any storage solution can be extended to have an HTTP interface by writing a wrapper in a general purpose programming language that exposes the database functionality.
However, this introduces additional overhead and complexity, as the database wrapper has to be implemented manually -- not to mention potential security issues.
\item [Free License]
The designed system shall be as widely applicable as possible and therefore not impose any financial requirements on the potential applicant.
Thus, the storage solution shall be available free for commercial use, e.g. in form of an open source license such as the GNU GPL\footnote{\url{https://www.gnu.org/licenses/gpl.html}} or Apache License\footnote{\url{https://www.apache.org/licenses/}}.
\end{description}

%I additionally rate storage solution candidates based on maturity, licensing, and applicability.
%Maturity is a rather subjective metric but can be based at least partly on the technology's age and its popularity.
%Licensing could be an issue if the technology has a proprietary license, or if modifications have to be made to the source code (although the latter case is rather unlikely).
%Applicability refers to how well the technology is presumed to fit to the use case of storing, accessing and modifying multiple related user actions.
%\todo{Be more specific - which scale?}

\subsection{Candidates}

Blabla

\begin{enumerate}
\item \ac{SQL} Databases: Blurb about SQL databases.
\begin{enumerate}
\item Postgres
\item MySQL
\item MS-SQL
\item Oracle
\item ...
\end{enumerate}
\item NoSQL Databases: Blurb about NoSQL databases.\cite{strauch2011nosql}
\begin{enumerate}
\item Document stores: Event Store, MongoDb
\item Graph databases: Neo4j
\item Column databases: Cassandra
\end{enumerate}
\item Other: Kafka+X (?)
\end{enumerate}

\subsection{Evaluation}

By analyzing each of the candidates, I determined that Event Store is indeed the most appropriate solution, Neo4j and MongoDb being the most promising alternatives.
Other solutions came short in at least one integral requirement.
See \cref{table:classifications:storage} for the complete results of the analysis.

Event Store is ... whatever.

All \ac{SQL} databases that I initially considered do not fulfill at least one of the mandatory requirements.
While Postgres\footnote{\url{https://www.postgresql.org/}} does not support sharding or other horizontal scaling solutions, MySQL\footnote{\url{https://www.mysql.com/}} and Oracle database\footnote{\url{https://www.oracle.com/database/}} have proprietary license models when used in a commercial context, which makes it difficult to recommend those databases in general.
Additionally, MySQL does not offer an own HTTP interface, which would complicate its implementation even more.
Cascading update is also problematic with \ac{SQL} databases: Although relations between tuples are possible, none of the considered databases have an inherent mechanism for modifying a series of dependent data entries.\todo{not precise enough}

Neo4j\footnote{\url{https://neo4j.com/}} appears to be a decent candidate.
As it is a graph database, it has a more explicit mapping of relations between tuples than \ac{SQL} databases.
This allows for concise update queries which can implement what was termed as "cascading updates" in the requirements.
Although Neo4j does not support clustering in the non-enterprise edition, its authors claim that the graph database nevertheless scales both vertically and horizontally~\cite{...}.
The full featured enterprise edition is not free for commercial use.

MongoDB\footnote{\url{https://www.mongodb.com/}} is another promising candidate.
It is arguably the most popular NoSQL database, especially for its great read and write performance~\cite{6625441,...} as well as scalability~\cite{...}.
MongoDb being a document database means that the data is more loosely coupled than, for example, in a graph database such as Neo4j.
This makes cascading updates harder to implement, but not impossible.
\ac{REST} support is already built-in.
In conclusion, MongoDb is, together with Event Store, the most suitable storage solution for this use case.

Key-value stores such as Apache Ignite\footnote{\url{https://ignite.apache.org/}} are not suitable for this use case.
One reason is that the concept of relations between values is in general not present in key-value stores.
Another reason is that their dynamic data structures do not support temporal queries and cascading updates out of the box.

Cassandra\footnote{\url{https://cassandra.apache.org/}} is strong in terms of performance and scalability.
Its \ac{CQL} is modeled closely to \ac{SQL}, which means that it has the same disadvantages in terms of cascading updates.
Another drawback is that it does not have a built-in HTTP interface.
Cassandra heavily emphasizes and excels in scalability and fault tolerance, but sacrifices read performance for this~\cite{6625441}.
This makes Cassandra a great choice for certain specialized use cases, but seems less suitable for a general purpose application.

\begin{table}[]
\centering
\caption{Classification of storage solutions}
\label{table:system:storage}
\begin{tabular}{lllllllllll}
\textbf{Name} & \textbf{Type} & \textbf{Casc. Updates} & \textbf{Sharding} & \textbf{HTTP} & \textbf{License} \\
Event Store & Document & yes & yes & yes & BSD \\
Postgres & SQL & no & no & no & PostgreSQL License \\
MySQL & SQL & no & yes & no & GPL / Proprietary \\
Oracle & SQL & no & yes & with addons & Proprietary \\
MongoDB & Document & ? & yes & yes & GNU AGPLv3 \\
Ignite & Key-Value & ? & yes & yes & Apache License v2 \\
Neo4j & Document & yes & no & yes & GPL / Proprietary \\
Cassandra & Document & no & yes & no & Apache License v2
\end{tabular}
\label{table:classifications:storage}
\end{table}

\section{Data Aggregation Solutions}
\label{sec:system:aggregation}

\subsection{Requirements}


\begin{description}
\item [HTTP interface]
\item [Aggregation over time]
\item [Aggregation over location]
\item [Free License]
\item [Performance and Scalability]
\end{description}

\subsection{Candidates}

\begin{enumerate}
\item Apache Lucene
\item Apache Solr
\item Elasticsearch
\item Apache Hadoop
\item Apache Spark
\item Native Event Store aggregation functions + custom webservice
\end{enumerate}

\subsection{Evaluation}

\section{Data Analysis Solutions}
\label{sec:system:analysis}

\subsection{Requirements}

\subsection{Candidates}

\begin{enumerate}
\item Kibana
\item Custom implementation
\item ?
\end{enumerate}

\subsection{Evaluation}