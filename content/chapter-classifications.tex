% !TEX root = ../thesis.tex
%
\chapter{Classifications}
\label{sec:classifications}

\section{Storage Solutions}
\label{sec:system:storage}

\subsection{Candidates}

\begin{enumerate}
\item SQL Databases: Blurb about SQL databases.
\begin{enumerate}
\item Postgres
\item MySQL
\item MS-SQL
\item Oracle
\item ...
\end{enumerate}
\item NoSQL Databases: Blurb about NoSQL databases.\cite{strauch2011nosql}
\begin{enumerate}
\item Document stores: Event Store, MongoDb
\item Key-value stores: Apache Ignite
\item Graph databases: Neo4j
\item Column databases: Cassandra
\end{enumerate}
\end{enumerate}
\subsection{Requirements}

\ac{CSE} requires the storage layer of such a system to have certain capabilities.
I have identified the following requirements for the storage solution, based on which the candidates will be evaluated:

\begin{description}
\item [Basic storage capabilities]
The storage solution must at least provide means of creating and reading persisted data records.
\item [Update, Undo \& Restore]
In order to facilitate experimentation, the storage solution shall have some means of undoing faulty data records.
This is desirable in cases where a change in the software introduced erroneous behaviour which has to be undone.
It would not be sufficient to just spin up a separate instance of the storage solution for the experiment because users should not be aware of whether they are part of the treatment group.
\item [Temporal queries]
It would presumably be useful to be able to retrieve all data records that were created within a specific time frame via some query.
These queries have to be efficient because running an analysis must not affect the storage solutions normal operation performance.
\item [Sharding]
For performance reasons, the storage solution shall support \emph{sharding}.
Sharding means that the data is split over multiple logical or physical storage instances, each of them shouldering a portion of the cumulative system load.
\item [REST interface]
The storage solution shall have a \ac{REST} HTTP interface such that it allows a web application to store and retrieve data.
\end{description}

I additionally rate storage solution candidates based on their maturity and licensing.
Maturity is a rather subjective metric but can be based at least partly on the technology's age and its popularity.
Licensing could be an issue if the technology has a proprietary license, or if modifications have to be made to the source code (although the latter case is rather unlikely).

\section{Data Aggregation Solutions}
\label{sec:system:aggregation}

\section{Data Analysis Solutions}
\label{sec:system:analysis}
