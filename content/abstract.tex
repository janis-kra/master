% !TEX root = ../thesis.tex
%
\pdfbookmark[0]{Abstract}{Abstract}
\chapter*{Abstract}
\label{sec:abstract}
\vspace*{-10mm}

An increasing amount of computer engineers have been frustrated with software products that fail due to the slow and inflexible phases of classical software engineering.
That is why alternative practices such as \emph{Continuous Software Engineering} have emerged and gained a lot of traction during the past decade.
In order to continuously improve a software system, reliable feedback from the user is needed.
%These experiments help the creators of a piece of software to test and validate their assumptions about a product and its features early in the development process.
%User feedback is the data upon which these experiments rely.
% What is the problem?
%The area of innovation via experimentation in the software industry is already well researched, but lacks a more holistic view on collecting, aggregating and analysing the passive user feedback needed for evaluating an experiment.
The area of Continuous Software Engineering is already well researched but lacks a more holistic view on collecting, aggregating and analyzing the user feedback needed for effectively improving the system.
%Why is this a problem?
%This is problematic because, although many companies realize the need for experimentation in their development process, they struggle with the systematic adoption of these practices.
This is problematic, because although many companies realize the need for continuity in their development process, they struggle with the systematic adoption of such practices.
%What is the / a solution?
By researching how to design and implement a system for establishing passive user feedback with considerably low effort, this thesis aims to fill this gap.
%It is expected that event sourcing is a good fit for realizing such a system, which will be further investigated in the thesis.
%Event sourcing is a storage solution in which data is stored in the form of immutable event objects representing actions on business objects; in combination, all these events represent the application state.
Designing and implementing said system includes evaluating software solutions that implement storing, aggregating, and analyzing user feedback data.
A prototype using the chosen technologies is then implemented.
%This involves setting up and configuring various services as well as writing software that connects them.
The system is evaluated by doing a user survey that compares passive user feedback collected from the test subjects via a customized web application with active user feedback gathered using a questionnaire.
Additional tests evaluate the performance of the developed system.
%This will presumably involve either a user test for gathering passive user feedback, or analysing the user actions in an existing data set.
%The gathered data is then aggregated and analysed using the implemented system, and then compared with the results of a questionnaire with usability experts.
%Why is this a solution / what is ?
Practitioners in the software industry could use the system presented in the thesis as a reference for introducing collection and analysis of passive user feedback into their software, which facilitates the adoption of Continuous Software Engineering.
